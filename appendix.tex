\appendix

\FloatBarrier
\chapter{Encoding Space Efficiency Analyses}

This appendix contains sample source code and corresponding machine code. We used the Compiler Explorer\footnote{\url{https://godbolt.org}~(visited on 14/07/2023)} to generate the machine code.

\FloatBarrier
\section{Bit Field Encoding Formats}
\label{sec:enc_formats}

In this section, we will present machine code generated for different types of bit field decoding. In general, shift-based bit field decoding generates smaller code compared to mask-based bit field decoding. Adjacent flag decoding is well-optimized across compilers and architectures.

\FloatBarrier
\subsection{Mask-Only Bit Field Decoding}
\label{sec:mask_dec}

In this section, we will present machine code generated for mask-only bit field decoding.

\begin{lstfloat}[!htb]
  \centering
  \caption{Mask-Only Bit Field Decoding.}
  \label{lst:mask_dec}
  \begin{lstlisting}[style=c]
int main(void) {
  //
  // Input: An address encoded as follows.
  // [Bits 11:0]  The offset into the memory page.
  // [Bits 31:12] The index of the memory page.
  //
  volatile unsigned int addr;
  //
  // Output: The base address of the memory page.
  //
  volatile unsigned int paddr = addr & 0xFFFFF000U;
  //
  // Output: The offset into the memory page.
  //
  volatile unsigned int poff = addr & 0x00000FFFU;
  return 0;
}
  \end{lstlisting}
\end{lstfloat}

\begin{lstfloat}[!htb]
  \centering
  \caption{Machine Code for \Cref{lst:mask_dec} (x86-64).}
  \label{lst:mask_dec_x86}
  \caption*{
    \begin{tabular}{r l}
      \textbf{Compiler:} & x64 msvc v19.33\\
      \textbf{Compiler Options:} & /O2\\
      \textbf{Code Size:} & 29 \glspl{byte}
    \end{tabular}
  }
  \begin{lstlisting}[style=asm]
0140001000    mov    eax, DWORD PTR [rsp+0x8]
0140001004    and    eax, 0xfffff000
0140001009    mov    DWORD PTR [rsp+0x8], eax
014000100d    mov    eax, DWORD PTR [rsp+0x8]
0140001011    and    eax, 0xfff
0140001016    mov    DWORD PTR [rsp+0x8], eax
014000101a    xor    eax, eax
014000101c    ret
014000101d
  \end{lstlisting}
\end{lstfloat}

\begin{lstfloat}[!htb]
  \centering
  \caption{Machine Code for \Cref{lst:mask_dec} (ARM64).}
  \label{lst:mask_dec_arm64}
  \caption*{
    \begin{tabular}{r l}
      \textbf{Compiler:} & ARM64 gcc 12.2\\
      \textbf{Compiler Options:} & -O3\\
      \textbf{Code Size:} & 40 \glspl{byte}
    \end{tabular}
  }
  \begin{lstlisting}[style=asm]
0000400540    sub    sp, sp, #0x10
0000400544    mov    w0, #0x0
0000400548    ldr    w1, [sp, #4]
000040054c    and    w1, w1, #0xfffff000
0000400550    str    w1, [sp, #8]
0000400554    ldr    w1, [sp, #4]
0000400558    and    w1, w1, #0xfff
000040055c    str    w1, [sp, #12]
0000400560    add    sp, sp, #0x10
0000400564    ret
0000400568
  \end{lstlisting}
\end{lstfloat}

\begin{lstfloat}[!htb]
  \centering
  \caption{Machine Code for \Cref{lst:mask_dec} (RISC-V).}
  \label{lst:mask_dec_riscv}
  \caption*{
    \begin{tabular}{r l}
      \textbf{Compiler:} & RISC-V rv64gc clang 15.0.0\\
      \textbf{Compiler Options:} & -O3\\
      \textbf{Code Size:} & 28 \glspl{byte}
    \end{tabular}
  }
  \begin{lstlisting}[style=asm]
0000000684    addi   sp, sp, -16
0000000686    lwu    a0, 12(sp)
000000068a    lui    a1, 0xfffff
000000068c    and    a0, a0, a1
000000068e    sw     a0, 8(sp)
0000000690    lwu    a0, 12(sp)
0000000694    slli   a0, a0, 0x34
0000000696    srli   a0, a0, 0x34
0000000698    sw     a0, 4(sp)
000000069a    li     a0, 0
000000069c    addi   sp, sp, 16
000000069e    ret
00000006a0
  \end{lstlisting}
\end{lstfloat}

\FloatBarrier
\subsection{Shift-Only Bit Field Decoding}
\label{sec:shift_dec}

In this section, we will present machine code generated for shift-only bit field decoding.

\begin{lstfloat}[!htb]
  \centering
  \caption{Shift-Only Bit Field Decoding.}
  \label{lst:shift_dec}
  \begin{lstlisting}[style=c]
int main(void) {
  //
  // Input: An address encoded as follows.
  // [Bits 19:0]  The index of the memory page.
  // [Bits 31:20] The offset into the memory page.
  //
  volatile unsigned int addr;
  //
  // Output: The base address of the memory page.
  //
  volatile unsigned int paddr = addr << 12U;
  //
  // Output: The offset into the memory page.
  //
  volatile unsigned int poff = addr >> 20U;
  return 0;
}
  \end{lstlisting}
\end{lstfloat}

\begin{lstfloat}[!htb]
  \centering
  \caption{Machine Code for \Cref{lst:shift_dec} (x86-64).}
  \label{lst:shift_dec_x86}
  \caption*{
    \begin{tabular}{r l}
      \textbf{Compiler:} & x64 msvc v19.33\\
      \textbf{Compiler Options:} & /O2\\
      \textbf{Code Size:} & 25 \glspl{byte}
    \end{tabular}
  }
  \begin{lstlisting}[style=asm]
0140001000    mov    eax, DWORD PTR [rsp+0x8]
0140001004    shl    eax, 0xc
0140001007    mov    DWORD PTR [rsp+0x8], eax
014000100b    mov    eax, DWORD PTR [rsp+0x8]
014000100f    shr    eax, 0x14
0140001012    mov    DWORD PTR [rsp+0x8], eax
0140001016    xor    eax, eax
0140001018    ret
0140001019
  \end{lstlisting}
\end{lstfloat}

\begin{lstfloat}[!htb]
  \centering
  \caption{Machine Code for \Cref{lst:shift_dec} (ARM64).}
  \label{lst:shift_dec_arm64}
  \caption*{
    \begin{tabular}{r l}
      \textbf{Compiler:} & ARM64 gcc 12.2\\
      \textbf{Compiler Options:} & -O3\\
      \textbf{Code Size:} & 40 \glspl{byte}
    \end{tabular}
  }
  \begin{lstlisting}[style=asm]
0000400540    sub    sp, sp, #0x10
0000400544    mov    w0, #0x0
0000400548    ldr    w1, [sp, #4]
000040054c    lsl    w1, w1, #12
0000400550    str    w1, [sp, #8]
0000400554    ldr    w1, [sp, #4]
0000400558    lsr    w1, w1, #20
000040055c    str    w1, [sp, #12]
0000400560    add    sp, sp, #0x10
0000400564    ret
0000400568
  \end{lstlisting}
\end{lstfloat}

\begin{lstfloat}[!htb]
  \centering
  \caption{Machine Code for \Cref{lst:shift_dec} (RISC-V).}
  \label{lst:shift_dec_riscv}
  \caption*{
    \begin{tabular}{r l}
      \textbf{Compiler:} & RISC-V rv64gc clang 15.0.0\\
      \textbf{Compiler Options:} & -O3\\
      \textbf{Code Size:} & 24 \glspl{byte}
    \end{tabular}
  }
  \begin{lstlisting}[style=asm]
0000000684    addi   sp, sp, -16
0000000686    lw     a0, 12(sp)
0000000688    slliw  a0, a0, 0xc
000000068c    sw     a0, 8(sp)
000000068e    lwu    a0, 12(sp)
0000000692    srli   a0, a0, 0x14
0000000694    sw     a0, 4(sp)
0000000696    li     a0, 0
0000000698    addi   sp, sp, 16
000000069a    ret
000000069c
  \end{lstlisting}
\end{lstfloat}

\clearpage
\FloatBarrier
\subsection{Flag Decoding and Compiler Optimizations}
\label{sec:flag_dec_opt}

In this section, we will present machine code generated for flag decoding.

\begin{lstfloat}[!htb]
  \centering
  \caption{Naive Bit Field Flag Decoding.}
  \label{lst:ff_dec}
  \begin{lstlisting}[style=c]
int main(void) {
  volatile unsigned int xyz;
  volatile unsigned int x = (xyz & 0x80000000U) != 0;
  volatile unsigned int y = (xyz & 0x40000000U) != 0;
  volatile unsigned int z = (xyz & 0x20000000U) != 0;
  return 0;
}
  \end{lstlisting}
\end{lstfloat}

\begin{lstfloat}[!htb]
  \centering
  \caption{Machine Code for \Cref{lst:ff_dec} (x86-64).}
  \caption*{
    \begin{tabular}{r l}
      \textbf{Compiler:} & x64 msvc v19.33\\
      \textbf{Compiler Options:} & /O2\\
      \textbf{Code Size:} & 38 \glspl{byte}
    \end{tabular}
  }
  \begin{lstlisting}[style=asm]
0000401000    push   ecx
0000401001    mov    eax, DWORD PTR [esp]
0000401004    shr    eax, 0x1f
0000401007    mov    DWORD PTR [esp], eax
000040100a    mov    eax, DWORD PTR [esp]
000040100d    shr    eax, 0x1e
0000401010    and    eax, 0x1
0000401013    mov    DWORD PTR [esp], eax
0000401016    mov    eax, DWORD PTR [esp]
0000401019    shr    eax, 0x1d
000040101c    and    eax, 0x1
000040101f    mov    DWORD PTR [esp], eax
0000401022    xor    eax, eax
0000401024    pop    ecx
0000401025    ret
0000401026
  \end{lstlisting}
\end{lstfloat}

\begin{lstfloat}[!htb]
  \centering
  \caption{Machine Code for \Cref{lst:ff_dec} (ARM64).}
  \caption*{
    \begin{tabular}{r l}
      \textbf{Compiler:} & ARM64 gcc 12.2\\
      \textbf{Compiler Options:} & -O3\\
      \textbf{Code Size:} & 52 \glspl{byte}
    \end{tabular}
  }
  \begin{lstlisting}[style=asm]
0000400540    sub    sp, sp, #0x10
0000400544    mov    w0, #0x0
0000400548    ldr    w1, [sp]
000040054c    lsr    w1, w1, #31
0000400550    str    w1, [sp, #4]
0000400554    ldr    w1, [sp]
0000400558    ubfx   x1, x1, #30, #1
000040055c    str    w1, [sp, #8]
0000400560    ldr    w1, [sp]
0000400564    ubfx   x1, x1, #29, #1
0000400568    str    w1, [sp, #12]
000040056c    add    sp, sp, #0x10
0000400570    ret
0000400574
  \end{lstlisting}
\end{lstfloat}

\begin{lstfloat}[!htb]
  \centering
  \caption{Machine Code for \Cref{lst:ff_dec} (RISC-V).}
  \caption*{
    \begin{tabular}{r l}
      \textbf{Compiler:} & RISC-V rv64gc clang 15.0.0\\
      \textbf{Compiler Options:} & -O3\\
      \textbf{Code Size:} & 32 \glspl{byte}
    \end{tabular}
  }
  \begin{lstlisting}[style=asm]
0000000684    addi   sp, sp, -16
0000000686    lwu    a0, 12(sp)
000000068a    srli   a0, a0, 0x1f
000000068c    sw     a0, 8(sp)
000000068e    lw     a0, 12(sp)
0000000690    slli   a0, a0, 0x21
0000000692    srli   a0, a0, 0x3f
0000000694    sw     a0, 4(sp)
0000000696    lw     a0, 12(sp)
0000000698    slli   a0, a0, 0x22
000000069a    srli   a0, a0, 0x3f
000000069c    sw     a0, 0(sp)
000000069e    li     a0, 0
00000006a0    addi   sp, sp, 16
00000006a2    ret
00000006a4
  \end{lstlisting}
\end{lstfloat}

\begin{lstfloat}[!htb]
  \centering
  \caption{Shift-Assisted Bit Field Flag Decoding.}
  \label{lst:ff_shift_dec}
  \begin{lstlisting}[style=c]
  int main(void) {
    volatile unsigned int xyz;
    unsigned int xyz2 = xyz >> 29U;
    volatile unsigned int x = (xyz2 & 0x04U) != 0;
    volatile unsigned int y = (xyz2 & 0x02U) != 0;
    volatile unsigned int z = (xyz2 & 0x01U) != 0;
    return 0;
  }
  \end{lstlisting}
\end{lstfloat}

\begin{lstfloat}[!htb]
  \centering
  \caption{Machine Code for \Cref{lst:ff_shift_dec} (x86-64).}
  \caption*{
    \begin{tabular}{r l}
      \textbf{Compiler:} & x64 msvc v19.33\\
      \textbf{Compiler Options:} & /O2\\
      \textbf{Code Size:} & 37 \glspl{byte}
    \end{tabular}
  }
  \begin{lstlisting}[style=asm]
0140001000    mov    ecx, DWORD PTR [rsp+0x8]
0140001004    shr    ecx, 0x1d
0140001007    mov    eax, ecx
0140001009    shr    eax, 0x2
014000100c    mov    DWORD PTR [rsp+0x8], eax
0140001010    mov    eax, ecx
0140001012    shr    eax, 1
0140001014    and    ecx, 0x1
0140001017    and    eax, 0x1
014000101a    mov    DWORD PTR [rsp+0x8], eax
014000101e    xor    eax, eax
0140001020    mov    DWORD PTR [rsp+0x8], ecx
0140001024    ret
0140001025
  \end{lstlisting}
\end{lstfloat}

\begin{lstfloat}[!htb]
  \centering
  \caption{Machine Code for \Cref{lst:ff_shift_dec} (ARM64).}
  \caption*{
    \begin{tabular}{r l}
      \textbf{Compiler:} & ARM64 gcc 12.2\\
      \textbf{Compiler Options:} & -O3\\
      \textbf{Code Size:} & 44 \glspl{byte}
    \end{tabular}
  }
  \begin{lstlisting}[style=asm]
0000400540    sub    sp, sp, #0x10
0000400544    mov    w0, #0x0
0000400548    ldr    w1, [sp]
000040054c    lsr    w2, w1, #31
0000400550    str    w2, [sp, #4]
0000400554    ubfx   x2, x1, #30, #1
0000400558    str    w2, [sp, #8]
000040055c    ubfx   x1, x1, #29, #1
0000400560    str    w1, [sp, #12]
0000400564    add    sp, sp, #0x10
0000400568    ret
000040056c
  \end{lstlisting}
\end{lstfloat}

\begin{lstfloat}[!htb]
  \centering
  \caption{Machine Code for \Cref{lst:ff_shift_dec} (RISC-V).}
  \caption*{
    \begin{tabular}{r l}
      \textbf{Compiler:} & RISC-V rv64gc clang 15.0.0\\
      \textbf{Compiler Options:} & -O3\\
      \textbf{Code Size:} & 32 \glspl{byte}
    \end{tabular}
  }
  \begin{lstlisting}[style=asm]
0000000684    addi   sp, sp, -16
0000000686    lwu    a0, 12(sp)
000000068a    srli   a1, a0, 0x1f
000000068e    sw     a1, 8(sp)
0000000690    slli   a1, a0, 0x21
0000000694    srli   a1, a1, 0x3f
0000000696    sw     a1, 4(sp)
0000000698    slli   a0, a0, 0x22
000000069a    srli   a0, a0, 0x3f
000000069c    sw     a0, 0(sp)
000000069e    li     a0, 0
00000006a0    addi   sp, sp, 16
00000006a2    ret
00000006a4
  \end{lstlisting}
\end{lstfloat}

\FloatBarrier
\section{Position Independence and Relative Addressing}

In this section, we will compare machine code generated for local non-preemptible \gls{image-symbol} references between \glspl{PIE} and non-\glspl{PIE}. Some cases of relative addressing are restricted to \glspl{PIE} --- non-\glspl{PIE} may resort to absolute addressing for performance reasons.

\begin{lstfloat}[!htb]
  \centering
  \caption{Local Non-Preemptible Image Symbol Reference.}
  \label{fig:local_sym}
  \begin{lstlisting}[style=c]
static volatile int var = 0;

static int __attribute__ ((noinline)) func(void) {
    return var;
}

int main(void) {
    return func();
}
  \end{lstlisting}
\end{lstfloat}

\begin{lstfloat}[!htb]
  \centering
  \caption{Machine Code for \Cref{fig:local_sym} (x86-64).}
  \caption*{
    \begin{tabular}{r l}
      \textbf{Compiler:} & x86-64 gcc 12.2\\
      \textbf{Left Compiler Options:} & -O3 -fno-pie -mcmodel=large -Wl,-no-pie\\
      \textbf{Right Compiler Options:} & -O3 -fPIE -Wl,-pie
    \end{tabular}
  }
  \begin{minipage}{0.4\textwidth}
    \begin{lstlisting}[style=asm]
func:
movabs  eax,ds:0x0
    R_X86_64_64
        .bss
ret    
main:
movabs  rax,0x0
    R_X86_64_64
        .text
jmp     rax
    \end{lstlisting}
  \end{minipage}\hfill
  \begin{minipage}{0.6\textwidth}
    \begin{lstlisting}[style=asm]
func:
mov     eax,DWORD PTR [rip+0x0]
    R_X86_64_PC32
        .bss-0x4
ret    
main:
jmp    5
    R_X86_64_PC32
        .text-0x4
    \end{lstlisting}
  \end{minipage}
\end{lstfloat}

\begin{lstfloat}[!htb]
  \centering
  \caption{Machine Code for \Cref{fig:local_sym} (ARM).}
  \caption*{
    \begin{tabular}{r l}
      \textbf{Compiler:} & ARM gcc 12.2\\
      \textbf{Left Compiler Options:} & -O3 -fno-pie -Wl,-no-pie\\
      \textbf{Right Compiler Options:} & -O3 -fPIE -Wl,-pie\\
    \end{tabular}
  }
  \begin{minipage}{0.5\textwidth}
    \begin{lstlisting}[style=asm]
func:
movw    r3, #0
    R_ARM_THM_MOVW_ABS_NC
        .LANCHOR0
movt    r3, #0
    R_ARM_THM_MOVT_ABS
        .LANCHOR0
ldr     r0, [r3, #0]
bx      lr
main:
b.w     0
    R_ARM_THM_JUMP24 func
\end{lstlisting}
  \end{minipage}\hfill
  \begin{minipage}{0.5\textwidth}
    \begin{lstlisting}[style=asm]
func:
ldr     r3, [pc, #4]
add     r3, pc
ldr     r0, [r3, #0]
bx      lr
.word   0x00000002
    R_ARM_REL32 .bss
main:
b.w     0
    R_ARM_THM_JUMP24 func
    \end{lstlisting}
  \end{minipage}
\end{lstfloat}

\begin{lstfloat}[!htb]
  \centering
  \caption{Machine Code for \Cref{fig:local_sym} (ARM64).}
  \caption*{
    \begin{tabular}{r l}
      \textbf{Compiler:} & ARM64 gcc 12.2\\
      \textbf{Compiler Options 1:} & -O3 -fno-pie -Wl,-no-pie\\
      \textbf{Compiler Options 2:} & -O3 -fPIE -Wl,-pie\\
    \end{tabular}\\
    Both generated identical code.
  }
  \begin{lstlisting}[style=asm]
func:
adrp    x0, 0
    R_AARCH64_ADR_PREL_PG_HI21 .bss
ldr     w0, [x0]
    R_AARCH64_LDST32_ABS_LO12_NC .bss
ret
main:
b       0
    R_AARCH64_JUMP26 .text
  \end{lstlisting}
\end{lstfloat}

\begin{lstfloat}[!htb]
  \centering
  \caption{Machine Code for \Cref{fig:local_sym} (RISC-V).}
  \caption*{
    \begin{tabular}{r l}
      \textbf{Compiler:} & RISC-V rv64gc gcc 12.2.0\\
      \textbf{Left Compiler Options:} & -O3 -fno-pie -Wl,-no-pie\\
      \textbf{Right Compiler Options:} & -O3 -fPIE -Wl,-pie
    \end{tabular}
  }
  \begin{minipage}{0.5\textwidth}
    \begin{lstlisting}[style=asm]
func:
lui     a5,0x0
    R_RISCV_HI20 var
    R_RISCV_RELAX *ABS*
lw      a0,0(a5)
    R_RISCV_LO12_I var
    R_RISCV_RELAX *ABS*
ret
main:
auipc   t1,0x0
    R_RISCV_CALL func
    R_RISCV_RELAX *ABS*
jr      t1
\end{lstlisting}
  \end{minipage}\hfill
  \begin{minipage}{0.5\textwidth}
    \begin{lstlisting}[style=asm]
func:
auipc   a0,0x0
    R_RISCV_PCREL_HI20
        .LANCHOR0
    R_RISCV_RELAX *ABS*
lw      a0,0(a0)
    R_RISCV_PCREL_LO12_I
        .L0 
    R_RISCV_RELAX *ABS*
ret
main:
auipc   t1,0x0
    R_RISCV_CALL func
    R_RISCV_RELAX *ABS*
jr      t1
    \end{lstlisting}
  \end{minipage}
\end{lstfloat}

\FloatBarrier
\chapter{UEFI Executable File Format (UE) Specification}
\label{chap:ue-spec}

This appendix contains a draft for the specification of the \glsxtrfull{UE} format~\cite{thesis-git}.

\FloatBarrier
\section{Basic Data Types}

\begin{table}[!htb]
  \centering
  \caption{Definition of the Basic Data Types.}
  \hfill\\
  \begin{tabular}{l c c}
    \toprule
    \textbf{Type} & \textbf{Size} [\gls{byte}] & \textbf{Max. Data Alignment} [\gls{byte}]\\
    \midrule
    \lstinline|uint8_t| & 1 & 1\\
    \lstinline|uint16_t| & 2 & 2\\
    \lstinline|uint32_t| & 4 & 4\\
    \lstinline|uint64_t| & 8 & 8\\
    \bottomrule
  \end{tabular}\\
  \hfill\\
  \hfill\\
  \emph{Note:} The \gls{data-alignment} requirement follows the \gls{natural-data-alignment} model, thus is to be understood as an upper bound, not an exact value.
\end{table}

\FloatBarrier
\section{UE Header}

\begin{table}[!htb]
  \centering
  \caption{Definition of the UE Machine Identifiers.}
  \hfill\\
  \begin{tabular}{c l}
    \toprule
    \textbf{Value} & \textbf{Description}\\
    \midrule
    0 & IA32\\
    1 & X64\\
    2 & ARM\\
    3 & AARCH64\\
    4 & RISCV32\\
    5 & RISCV64\\
    6 & RISCV128\\
    \bottomrule
  \end{tabular}
\end{table}

\begin{table}[!htb]
  \centering
  \caption{Definition of the UE Subsystem Identifiers.}
  \hfill\\
  \begin{tabular}{c l}
    \toprule
    \textbf{Value} & \textbf{Description}\\
    \midrule
    0 & Application\\
    1 & Boot Services Driver\\
    2 & Runtime Driver\\
    \bottomrule
  \end{tabular}
\end{table}

\begin{table}[!htb]
  \centering
  \caption{Definition of the UE Header (\lstinline|UE\_HEADER|).}
  \hfill\\
  \begin{tabularx}{\textwidth}{l X}
    \toprule
    \textbf{Type} & \textbf{Description}\\
    \midrule
    \lstinline|uint16_t| & The \gls{file-magic-number} to identify the \glsxtrshort{UE} file format. Must be `UE'.\\
    \midrule
    \lstinline|uint8_t| & Information about the \gls{image} kind and supported architectures.\\
    Bits 2:0 & Indicates the subsystem.\\
    Bits 7:3 & Indicates the supported architectures.\\
    \midrule
    \lstinline|uint8_t| & Information about the \glsxtrshort{UE} load tables and segments.\\
    Bits 2:0 & The number of \glsxtrshort{UE} load tables.\\
    Bits 7:3 & The index of the last segment in the \glsxtrshort{UE} segment table.\\
    \midrule
    \lstinline|uint32_t| & Indicates the offset of the \glsxtrshort{UE} entry point in the \glsxtrshort{UE} \gls{address-space}.\\
    \midrule
    \lstinline|uint64_t| & Information about the \glsxtrshort{UE} \gls{image}.\\
    Bits 51:0  & The base UEFI page of the \glsxtrshort{UE} \gls{image}, i.e., the base address in 4 KiB units.\\
    Bits 56:52 & Reserved for future use. Must be zero.\\
    Bit 57     & Indicates whether the \glsxtrshort{UE} \gls{image} is designated for a fixed address.\\
    Bit 58     & Indicates whether the \glsxtrshort{UE} relocation table has been stripped.\\
    Bit 59     & Indicates whether \glsxtrshort{UE} chained fixups are used.\\
    Bits 63:60 & The shift exponent, offset by $-12$, for the \glsxtrshort{UE} segment alignment in \glspl{byte}.\\
    \midrule
    \lstinline|UE\_SEGMENT\[\]| & The \glsxtrshort{UE} segment table. It contains all data of the \glsxtrshort{UE} address space.\\
    \midrule
    \lstinline|UE\_LOAD\_TABLE\[\]| & The \glsxtrshort{UE} load tables. They contain data useful for UE loading.\\
    \bottomrule
  \end{tabularx}
\end{table}

\FloatBarrier
\section{UE Segment Table}

All \glsxtrshort{UE} segments are contiguous in the \glsxtrshort{UE} \gls{address-space}. The offset of the first \glsxtrshort{UE} segment in the \glsxtrshort{UE} \gls{address-space} is $0$. All \glsxtrshort{UE} segments' data are contiguous in the \glsxtrshort{UE} file. The offset of the first \glsxtrshort{UE} segment in the \glsxtrshort{UE} file is the end of the \glsxtrshort{UE} file header.

\begin{table}[!htb]
  \centering
  \caption{Definition of the UE Segment Permissions Configurations.}
  \hfill\\
  \begin{tabular}{c l}
    \toprule
    \textbf{Value} & \textbf{Description}\\
    \midrule
    0 & Execute\\
    1 & Read, Execute\\
    2 & Read, Write\\
    3 & Read\\
    \bottomrule
  \end{tabular}
\end{table}

\begin{table}[!htb]
  \centering
  \caption{Definition of a UE Segment Header (\lstinline|UE\_SEGMENT|).}
  \hfill\\
  \begin{tabularx}{\textwidth}{c X}
    \toprule
    \textbf{Type} & \textbf{Description}\\
    \midrule
    \lstinline|uint32_t| & Information about the \glsxtrshort{UE} segment in the \glsxtrshort{UE} \gls{address-space}.\\
    Bits 19:0 & The size, in 4-KiB units, of the \glsxtrshort{UE} segment in the \glsxtrshort{UE} \gls{address-space}.\\
    Bits 21:20 & The \glsxtrshort{UE} segment permissions.\\
    Bits 31:22 & Reserved for future use. Must be zero.\\
    \midrule
    \lstinline|uint32_t| & The size, in \glspl{byte}, of the \glsxtrshort{UE} segment in the \glsxtrshort{UE} file.\\
    \bottomrule
  \end{tabularx}
\end{table}

\FloatBarrier
\section{UE Load Tables}

All \glsxtrshort{UE} load tables are contiguous in the \glsxtrshort{UE} file. The offset of the first \glsxtrshort{UE} load table in the \glsxtrshort{UE} file is the end of the last \glsxtrshort{UE} segment in the \glsxtrshort{UE} file. All \glsxtrshort{UE} load tables are ordered ascending by their identifier.

\begin{table}[!htb]
  \centering
  \caption{Definition of the UE Load Table Identifiers.}
  \hfill\\
  \begin{tabular}{c l}
    \toprule
    \textbf{Value} & \textbf{Description}\\
    \midrule
    0 & Relocation Table\\
    1 & Debug Table\\
    \bottomrule
  \end{tabular}
\end{table}

\begin{table}[!htb]
  \centering
  \caption{Definition of a UE Load Table Header (\lstinline|UE\_LOAD\_TABLE|).}
  \hfill\\
  \begin{tabularx}{\textwidth}{l X}
    \toprule
    \textbf{Type} & \textbf{Description}\\
    \midrule
    \lstinline|uint32_t| & Information about the \glsxtrshort{UE} load table.\\
    Bits 28:0 & The size, in $8$-\gls{byte} units, of the \glsxtrshort{UE} load table in the \glsxtrshort{UE} file.\\
    Bits 31:29 & The identifier of the \glsxtrshort{UE} load table.\\
    \bottomrule
  \end{tabularx}
\end{table}

\FloatBarrier
\section{UE Relocation Load Table}

\begin{table}[!htb]
  \centering
  \caption{Definition of the Generic UE Relocation Identifiers.}
  \hfill\\
  \begin{tabular}{c l}
    \toprule
    \textbf{Value} & \textbf{Description}\\
    \midrule
    0 & Absolute 32-bit\\
    1 & Absolute 64-bit\\
    \bottomrule
  \end{tabular}
\end{table}

\begin{table}[!htb]
  \centering
  \caption{Definition of the Common Header of UE Chained Relocation Fixups (\lstinline|UE\_RELOC\_FIXUP\_HDR|).}
  \hfill\\
  \begin{tabularx}{\textwidth}{l X}
    \toprule
    \textbf{Type} & \textbf{Description}\\
    \midrule
    \lstinline|uint16_t| & \\
    Bits 3:0 & The relocation type of the next chained relocation fixup. Only valid when [Bits 15:4] are not \texttt{0x0FFF}.\\
    Bits 15:4 & The offset to the next chained relocation fixup from the end of the current one. If \texttt{0x0FFF}, the current chain is terminated. Consult the fixup root for further relocation fixups.\\
    \bottomrule
  \end{tabularx}
\end{table}

\begin{table}[!htb]
  \centering
  \caption{Definition of the Generic 64-Bit UE Chained Relocation Fixup (\lstinline|UE\_RELOC\_FIXUP\_64|).}
  \hfill\\
  \begin{tabularx}{\textwidth}{l X}
    \toprule
    \textbf{Type} & \textbf{Description}\\
    \midrule
    \lstinline|uint64_t| & \\
    Bits 15:0 & The common header of UE chained relocation fixups.\\
    Bits 47:16 & The address value to relocate.\\
    Bits 63:48 & Must be zero.\\
    \bottomrule
  \end{tabularx}
\end{table}

\begin{table}[!htb]
  \centering
  \caption{Definition of a UE Fixup Root (\lstinline|UE\_FIXUP\_ROOT|).}
  \hfill\\
  \begin{tabularx}{\textwidth}{l X}
    \toprule
    \textbf{Type} & \textbf{Description}\\
    \midrule
    \lstinline|uint32_t| & The offset of the first head fixup, in \glspl{byte}, from the end of the previous \glsxtrshort{UE} relocation fixup (chained or not). The first \glsxtrshort{UE} fixup root is relative to $0$.\\
    \midrule
    \lstinline|uint16_t[]| & The head fixups of the \glsxtrshort{UE} fixup root.\\
    Bits 3:0 & The type of the \glsxtrshort{UE} relocation fixup.\\
    Bits 15:4 & The offset of the next \glsxtrshort{UE} head fixup from the end of the last UE relocation fixup in the chain (if chained). If \texttt{0x0FFF}, the current fixup root is terminated.\\
    \bottomrule
  \end{tabularx}
\end{table}

\FloatBarrier
\section{UE Debug Load Table}

\begin{table}[!htb]
  \centering
  \caption{Definition of the UE Debug Table Header (\lstinline|UE\_DEBUG\_TABLE|).}
  \hfill\\
  \begin{tabularx}{\textwidth}{l X}
    \toprule
    \textbf{Type} & \textbf{Description}\\
    \midrule
    \lstinline|uint8_t| & Information about the \gls{image} regarding the symbols file.\\
    Bits 1:0 & The offset, in \gls{image} alignment units, to be subtracted from the \glsxtrshort{UE} base address in order to retrieve the \glsxtrshort{UE} symbols base address.\\
    Bits 7:2 & Reserved for future use. Must be zero.\\
    \midrule
    \lstinline|uint8_t| & The length, in \glspl{byte}, of the \glsxtrshort{UE} symbols path (excluding the terminator).\\
    \midrule
    \lstinline|uint8_t[]| & The \glsxtrshort{UE} symbols path. Must be \textbackslash0-terminated.\\
    \midrule
    \lstinline|uint8_t[8][]| & The \glsxtrshort{UE} segment name table. Each entry must be \textbackslash0-terminated. The order matches the \glsxtrshort{UE} segment table.\\
    \bottomrule
  \end{tabularx}
\end{table}

\FloatBarrier
\chapter{UE File Format Space-Saving}
\label{chap:space_savings}

This appendix contains the space-saving reports for converting all modules of \gls{OVMF} and \gls{ArmVirtQemu} from \glsxtrshort{PE} to \glsxtrshort{UE}. All modules have been compiled in \texttt{RELEASE} mode. It was generated by a custom script and is publicly available and reproducible for artifact evaluation~\cite{thesis-git}.

\begin{longtable}{l c c c}
  \caption{OVMF (IA32, GCC5) PE-to-UE Size Comparison.}\\\\
  \toprule
  \multirow{2}{*}[-2pt]{\textbf{Module}} & \multicolumn{2}{c}{\textbf{Size} [KiB]} & \multirow{2}{*}[-2pt]{\textbf{Saving} [\%]}\\
  \cmidrule{2-3}
  & \textbf{PE} & \textbf{UE} &\\
  \midrule
  StatusCodeHandlerPei & 1.25 & 0.62 & 50.00\\
  WatchdogTimer & 1.44 & 0.75 & 47.83\\
  DpcDxe & 1.44 & 0.85 & 40.76\\
  IncompatiblePciDeviceSupportDxe & 1.62 & 1.01 & 37.98\\
  MonotonicCounterRuntimeDxe & 1.75 & 1.09 & 37.50\\
  ReportStatusCodeRouterPei & 1.81 & 1.14 & 37.07\\
  CapsuleRuntimeDxe & 1.75 & 1.10 & 37.05\\
  Metronome & 1.75 & 1.11 & 36.61\\
  NullMemoryTestDxe & 1.75 & 1.14 & 34.82\\
  StatusCodeHandlerRuntimeDxe & 1.81 & 1.23 & 31.90\\
  EnglishDxe & 2.00 & 1.40 & 30.08\\
  CpuIo2Dxe & 2.25 & 1.65 & 26.74\\
  SecurityStubDxe & 2.50 & 1.87 & 25.31\\
  EmuVariableFvbRuntimeDxe & 2.88 & 2.24 & 22.01\\
  BootGraphicsResourceTableDxe & 2.50 & 1.95 & 21.88\\
  LocalApicTimerDxe & 2.62 & 2.07 & 21.13\\
  ReportStatusCodeRouterRuntimeDxe & 3.12 & 2.52 & 19.50\\
  ResetSystemRuntimeDxe & 3.19 & 2.58 & 19.12\\
  VirtioPciDeviceDxe & 3.19 & 2.59 & 18.87\\
  SataController & 3.69 & 3.05 & 17.16\\
  SmbiosPlatformDxe & 3.50 & 2.92 & 16.52\\
  IoMmuDxe & 4.00 & 3.36 & 16.02\\
  VirtioRngDxe & 3.81 & 3.21 & 15.78\\
  SioBusDxe & 4.06 & 3.44 & 15.38\\
  S3Resume2Pei & 4.62 & 3.95 & 14.53\\
  FvbServicesRuntimeDxe & 4.50 & 3.88 & 13.89\\
  QemuKernelLoaderFsDxe & 5.12 & 4.41 & 13.87\\
  RuntimeDxe & 4.44 & 3.84 & 13.56\\
  ConPlatformDxe & 5.38 & 4.72 & 12.21\\
  VirtioBlkDxe & 4.75 & 4.20 & 11.51\\
  QemuRamfbDxe & 6.12 & 5.46 & 10.84\\
  PcRtc & 6.94 & 6.22 & 10.36\\
  VirtioScsiDxe & 6.19 & 5.56 & 10.10\\
  DiskIoDxe & 6.06 & 5.47 & 9.79\\
  PcdPeim & 6.88 & 6.23 & 9.32\\
  Virtio10 & 7.19 & 6.55 & 8.91\\
  ScsiBus & 6.44 & 5.87 & 8.86\\
  DxeIpl & 6.69 & 6.11 & 8.64\\
  SmbiosDxe & 7.88 & 7.21 & 8.43\\
  UsbMassStorageDxe & 8.69 & 8.00 & 7.91\\
  PciHotPlugInitDxe & 7.12 & 6.56 & 7.89\\
  ArpDxe & 9.69 & 9.00 & 7.10\\
  PcdDxe & 8.69 & 8.11 & 6.65\\
  FaultTolerantWriteDxe & 9.38 & 8.76 & 6.58\\
  QemuFwCfgAcpiPlatform & 9.06 & 8.49 & 6.29\\
  Ps2KeyboardDxe & 10.75 & 10.09 & 6.18\\
  S3SaveStateDxe & 10.38 & 9.73 & 6.17\\
  VirtioNetDxe & 9.50 & 8.93 & 6.00\\
  AtaBusDxe & 11.12 & 10.54 & 5.27\\
  SecMain & 12.88 & 12.20 & 5.22\\
  PartitionDxe & 12.94 & 12.27 & 5.19\\
  UdfDxe & 13.31 & 12.64 & 5.05\\
  DriverHealthManagerDxe & 14.19 & 13.49 & 4.90\\
  GraphicsConsoleDxe & 13.06 & 12.44 & 4.78\\
  UhciDxe & 12.31 & 11.73 & 4.76\\
  PlatformDxe & 13.31 & 12.68 & 4.75\\
  LogoDxe & 12.88 & 12.27 & 4.67\\
  VirtioGpuDxe & 13.06 & 12.45 & 4.67\\
  UsbKbDxe & 12.50 & 11.92 & 4.62\\
  EbcDxe & 13.31 & 12.71 & 4.52\\
  AcpiTableDxe & 15.44 & 14.76 & 4.40\\
  TerminalDxe & 15.62 & 14.95 & 4.35\\
  VlanConfigDxe & 16.62 & 15.91 & 4.32\\
  ConSplitterDxe & 16.12 & 15.43 & 4.31\\
  EhciDxe & 14.31 & 13.70 & 4.26\\
  SnpDxe & 17.75 & 17.02 & 4.09\\
  PciSioSerialDxe & 15.94 & 15.29 & 4.07\\
  UsbBusDxe & 14.19 & 13.62 & 4.02\\
  QemuVideoDxe & 15.62 & 15.02 & 3.90\\
  PciHostBridgeDxe & 19.62 & 18.95 & 3.42\\
  LinuxInitrdDynamicShellCommand & 19.06 & 18.41 & 3.40\\
  MnpDxe & 20.62 & 19.95 & 3.26\\
  Fat & 20.44 & 19.78 & 3.21\\
  AtaAtapiPassThruDxe & 23.50 & 22.76 & 3.16\\
  NvmExpressDxe & 22.44 & 21.73 & 3.13\\
  Udp4Dxe & 21.19 & 20.53 & 3.10\\
  Dhcp4Dxe & 23.50 & 22.78 & 3.06\\
  PlatformPei & 22.06 & 21.41 & 2.97\\
  VirtioFsDxe & 20.00 & 19.41 & 2.97\\
  Mtftp4Dxe & 23.31 & 22.68 & 2.71\\
  PeiCore & 23.19 & 22.57 & 2.66\\
  VariableRuntimeDxe & 26.56 & 25.87 & 2.62\\
  BootScriptExecutorDxe & 26.81 & 26.13 & 2.53\\
  CpuMpPei & 27.69 & 26.99 & 2.51\\
  XhciDxe & 23.69 & 23.09 & 2.51\\
  tftpDynamicCommand & 27.75 & 27.08 & 2.42\\
  RamDiskDxe & 26.06 & 25.44 & 2.40\\
  ScsiDisk & 27.06 & 26.41 & 2.40\\
  DevicePathDxe & 31.94 & 31.27 & 2.08\\
  TcpDxe & 37.50 & 36.73 & 2.04\\
  httpDynamicCommand & 35.19 & 34.48 & 2.00\\
  PciBusDxe & 36.75 & 36.10 & 1.76\\
  UefiPxeBcDxe & 50.06 & 49.31 & 1.50\\
  Ip4Dxe & 51.88 & 51.15 & 1.40\\
  SetupBrowser & 58.38 & 57.59 & 1.34\\
  DisplayEngine & 54.12 & 53.42 & 1.30\\
  BdsDxe & 62.62 & 61.84 & 1.26\\
  HiiDatabase & 73.31 & 72.56 & 1.02\\
  IScsiDxe & 90.25 & 89.48 & 0.86\\
  UiApp & 98.06 & 97.30 & 0.78\\
  DxeCore & 117.69 & 116.81 & 0.74\\
  Shell & 768.88 & 767.53 & 0.17\\
  \midrule
  \textbf{Average} & 24.90 & 24.24 & 10.50\\
  \textbf{Total} & 2,539.81 & 2,472.85 & 2.64\\
  \bottomrule
\end{longtable}

\begin{longtable}{l c c c}
  \caption{OVMF (IA32, CLANGDWARF) PE-to-UE Size Comparison.}\\\\
  \toprule
  \multirow{2}{*}[-2pt]{\textbf{Module}} & \multicolumn{2}{c}{\textbf{Size} [KiB]} & \multirow{2}{*}[-2pt]{\textbf{Saving} [\%]}\\
  \cmidrule{2-3}
  & \textbf{PE} & \textbf{UE} &\\
  \midrule
  StatusCodeHandlerPei & 1.25 & 0.58 & 53.75\\
  WatchdogTimer & 1.44 & 0.77 & 46.74\\
  DpcDxe & 1.44 & 0.83 & 42.39\\
  CapsuleRuntimeDxe & 1.62 & 0.98 & 39.42\\
  ReportStatusCodeRouterPei & 1.69 & 1.05 & 37.96\\
  NullMemoryTestDxe & 1.75 & 1.10 & 37.05\\
  IncompatiblePciDeviceSupportDxe & 1.56 & 0.98 & 37.00\\
  MonotonicCounterRuntimeDxe & 1.75 & 1.12 & 36.16\\
  Metronome & 1.50 & 0.98 & 34.90\\
  StatusCodeHandlerRuntimeDxe & 1.75 & 1.20 & 31.70\\
  EnglishDxe & 1.94 & 1.33 & 31.45\\
  CpuIo2Dxe & 2.25 & 1.62 & 27.78\\
  SecurityStubDxe & 2.50 & 1.87 & 25.31\\
  BootGraphicsResourceTableDxe & 2.56 & 1.94 & 24.39\\
  LocalApicTimerDxe & 2.50 & 1.90 & 24.06\\
  EmuVariableFvbRuntimeDxe & 2.75 & 2.11 & 23.30\\
  ResetSystemRuntimeDxe & 3.00 & 2.36 & 21.35\\
  ReportStatusCodeRouterRuntimeDxe & 3.00 & 2.43 & 19.01\\
  SmbiosPlatformDxe & 3.44 & 2.79 & 18.86\\
  VirtioPciDeviceDxe & 3.31 & 2.73 & 17.45\\
  IoMmuDxe & 3.69 & 3.05 & 17.16\\
  VirtioRngDxe & 3.94 & 3.28 & 16.67\\
  SataController & 3.69 & 3.11 & 15.68\\
  S3Resume2Pei & 4.06 & 3.44 & 15.38\\
  SioBusDxe & 4.06 & 3.46 & 14.81\\
  QemuKernelLoaderFsDxe & 4.81 & 4.14 & 13.96\\
  VirtioBlkDxe & 5.00 & 4.33 & 13.44\\
  RuntimeDxe & 4.31 & 3.76 & 12.86\\
  FvbServicesRuntimeDxe & 4.44 & 3.87 & 12.85\\
  ConPlatformDxe & 5.56 & 4.95 & 11.10\\
  QemuRamfbDxe & 6.12 & 5.45 & 10.97\\
  VirtioScsiDxe & 6.19 & 5.55 & 10.23\\
  PcdPeim & 6.62 & 5.97 & 9.91\\
  DxeIpl & 6.25 & 5.63 & 9.88\\
  ScsiBus & 6.31 & 5.71 & 9.53\\
  DiskIoDxe & 6.44 & 5.83 & 9.47\\
  PcRtc & 6.25 & 5.67 & 9.25\\
  Virtio10 & 7.12 & 6.48 & 9.10\\
  PciHotPlugInitDxe & 6.31 & 5.76 & 8.79\\
  SmbiosDxe & 7.81 & 7.18 & 8.10\\
  UsbMassStorageDxe & 9.12 & 8.41 & 7.88\\
  PcdDxe & 8.69 & 8.05 & 7.28\\
  QemuFwCfgAcpiPlatform & 8.94 & 8.35 & 6.56\\
  S3SaveStateDxe & 10.25 & 9.58 & 6.55\\
  ArpDxe & 9.88 & 9.26 & 6.25\\
  VirtioNetDxe & 9.44 & 8.85 & 6.21\\
  FaultTolerantWriteDxe & 9.44 & 8.85 & 6.21\\
  Ps2KeyboardDxe & 10.25 & 9.66 & 5.79\\
  AtaBusDxe & 11.12 & 10.51 & 5.55\\
  DriverHealthManagerDxe & 14.31 & 13.58 & 5.13\\
  PartitionDxe & 12.44 & 11.82 & 4.96\\
  GraphicsConsoleDxe & 13.50 & 12.84 & 4.86\\
  VirtioGpuDxe & 13.12 & 12.49 & 4.82\\
  UhciDxe & 12.56 & 11.96 & 4.79\\
  AcpiTableDxe & 15.19 & 14.46 & 4.78\\
  UsbBusDxe & 14.44 & 13.75 & 4.76\\
  EbcDxe & 12.50 & 11.91 & 4.69\\
  PlatformDxe & 13.44 & 12.81 & 4.65\\
  UdfDxe & 13.31 & 12.72 & 4.46\\
  LogoDxe & 12.81 & 12.24 & 4.45\\
  UsbKbDxe & 12.56 & 12.01 & 4.42\\
  PciSioSerialDxe & 16.06 & 15.37 & 4.33\\
  ConSplitterDxe & 16.25 & 15.56 & 4.23\\
  EhciDxe & 14.69 & 14.07 & 4.20\\
  TerminalDxe & 15.88 & 15.22 & 4.13\\
  QemuVideoDxe & 16.56 & 15.92 & 3.87\\
  SnpDxe & 17.62 & 16.95 & 3.86\\
  VlanConfigDxe & 16.62 & 16.00 & 3.76\\
  PciHostBridgeDxe & 18.62 & 17.95 & 3.65\\
  LinuxInitrdDynamicShellCommand & 17.94 & 17.31 & 3.48\\
  PlatformPei & 19.62 & 18.95 & 3.46\\
  Fat & 20.38 & 19.71 & 3.26\\
  MnpDxe & 21.38 & 20.71 & 3.11\\
  AtaAtapiPassThruDxe & 23.44 & 22.72 & 3.07\\
  NvmExpressDxe & 22.75 & 22.06 & 3.02\\
  Udp4Dxe & 21.06 & 20.43 & 3.00\\
  VirtioFsDxe & 20.94 & 20.34 & 2.87\\
  XhciDxe & 24.56 & 23.88 & 2.77\\
  Mtftp4Dxe & 23.56 & 22.92 & 2.72\\
  Dhcp4Dxe & 23.31 & 22.70 & 2.65\\
  RamDiskDxe & 26.62 & 25.92 & 2.64\\
  PeiCore & 25.75 & 25.08 & 2.61\\
  VariableRuntimeDxe & 27.19 & 26.50 & 2.53\\
  SecMain & 25.31 & 24.68 & 2.50\\
  tftpDynamicCommand & 27.31 & 26.64 & 2.46\\
  CpuMpPei & 26.62 & 25.99 & 2.38\\
  BootScriptExecutorDxe & 30.12 & 29.41 & 2.36\\
  ScsiDisk & 28.50 & 27.87 & 2.22\\
  DevicePathDxe & 32.56 & 31.89 & 2.06\\
  httpDynamicCommand & 35.06 & 34.34 & 2.05\\
  PciBusDxe & 37.00 & 36.34 & 1.77\\
  TcpDxe & 38.56 & 37.90 & 1.72\\
  Ip4Dxe & 51.69 & 50.93 & 1.47\\
  UefiPxeBcDxe & 50.12 & 49.48 & 1.29\\
  SetupBrowser & 60.56 & 59.78 & 1.29\\
  DisplayEngine & 57.88 & 57.16 & 1.23\\
  BdsDxe & 63.31 & 62.56 & 1.18\\
  HiiDatabase & 76.69 & 75.86 & 1.08\\
  IScsiDxe & 91.06 & 90.27 & 0.87\\
  UiApp & 99.81 & 98.99 & 0.82\\
  DxeCore & 108.81 & 108.00 & 0.75\\
  Shell & 772.50 & 771.15 & 0.17\\
  \midrule
  \textbf{Average} & 25.13 & 24.48 & 10.66\\
  \textbf{Total} & 2,563.62 & 2,496.91 & 2.60\\
  \bottomrule
\end{longtable}

\begin{longtable}{l c c c}
  \caption{OVMF (X64, GCC5) PE-to-UE Size Comparison.}\\\\
  \toprule
  \multirow{2}{*}[-2pt]{\textbf{Module}} & \multicolumn{2}{c}{\textbf{Size} [KiB]} & \multirow{2}{*}[-2pt]{\textbf{Saving} [\%]}\\
  \cmidrule{2-3}
  & \textbf{PE} & \textbf{UE} &\\
  \midrule
  StatusCodeHandlerPei & 1.12 & 0.59 & 47.92\\
  WatchdogTimer & 1.44 & 0.79 & 45.11\\
  DpcDxe & 1.56 & 0.93 & 40.50\\
  NullMemoryTestDxe & 1.69 & 1.02 & 39.35\\
  MonotonicCounterRuntimeDxe & 1.62 & 0.99 & 38.94\\
  ReportStatusCodeRouterPei & 1.88 & 1.24 & 33.75\\
  IncompatiblePciDeviceSupportDxe & 1.81 & 1.22 & 32.76\\
  EnglishDxe & 2.25 & 1.54 & 31.60\\
  CapsuleRuntimeDxe & 1.69 & 1.16 & 31.02\\
  StatusCodeHandlerRuntimeDxe & 1.75 & 1.23 & 29.46\\
  Metronome & 2.25 & 1.61 & 28.47\\
  SecurityStubDxe & 2.88 & 2.20 & 23.37\\
  BootGraphicsResourceTableDxe & 2.88 & 2.24 & 22.01\\
  EmuVariableFvbRuntimeDxe & 2.81 & 2.25 & 20.00\\
  VirtioPciDeviceDxe & 3.75 & 3.09 & 17.71\\
  VirtioRngDxe & 4.31 & 3.57 & 17.21\\
  CpuIo2Dxe & 4.25 & 3.54 & 16.73\\
  LocalApicTimerDxe & 3.75 & 3.12 & 16.67\\
  ReportStatusCodeRouterRuntimeDxe & 3.75 & 3.13 & 16.46\\
  SioBusDxe & 4.69 & 3.96 & 15.50\\
  SataController & 4.19 & 3.55 & 15.11\\
  ResetSystemRuntimeDxe & 4.31 & 3.68 & 14.67\\
  RuntimeDxe & 4.44 & 3.79 & 14.61\\
  VirtioBlkDxe & 5.25 & 4.57 & 12.95\\
  SmbiosPlatformDxe & 4.94 & 4.34 & 12.18\\
  QemuKernelLoaderFsDxe & 6.31 & 5.56 & 11.88\\
  AmdSevDxe & 5.56 & 4.95 & 11.10\\
  ConPlatformDxe & 6.25 & 5.59 & 10.62\\
  S3Resume2Pei & 6.69 & 6.01 & 10.16\\
  VirtioScsiDxe & 6.81 & 6.13 & 9.98\\
  DiskIoDxe & 6.69 & 6.05 & 9.58\\
  TdxDxe & 6.50 & 5.88 & 9.50\\
  PcdPeim & 7.88 & 7.16 & 9.13\\
  ScsiBus & 6.94 & 6.30 & 9.12\\
  DxeIpl & 7.38 & 6.72 & 8.90\\
  Virtio10 & 8.25 & 7.53 & 8.71\\
  FvbServicesRuntimeDxe & 7.19 & 6.57 & 8.59\\
  QemuRamfbDxe & 7.88 & 7.20 & 8.53\\
  PcRtc & 7.94 & 7.31 & 7.87\\
  SmbiosDxe & 9.38 & 8.73 & 6.92\\
  UsbMassStorageDxe & 10.50 & 9.78 & 6.85\\
  PcdDxe & 9.81 & 9.16 & 6.69\\
  IoMmuDxe & 9.75 & 9.12 & 6.41\\
  VirtioNetDxe & 10.62 & 9.95 & 6.40\\
  S3SaveStateDxe & 12.12 & 11.37 & 6.25\\
  EbcDxe & 12.81 & 12.03 & 6.10\\
  ArpDxe & 11.38 & 10.69 & 6.04\\
  PciHotPlugInitDxe & 8.56 & 8.07 & 5.75\\
  Ps2KeyboardDxe & 12.56 & 11.88 & 5.41\\
  FaultTolerantWriteDxe & 10.19 & 9.64 & 5.37\\
  UdfDxe & 13.50 & 12.78 & 5.32\\
  AtaBusDxe & 13.00 & 12.32 & 5.23\\
  GraphicsConsoleDxe & 14.50 & 13.78 & 4.96\\
  VirtioGpuDxe & 14.19 & 13.48 & 4.96\\
  QemuFwCfgAcpiPlatform & 10.62 & 10.10 & 4.93\\
  LogoDxe & 12.94 & 12.30 & 4.89\\
  PartitionDxe & 13.19 & 12.55 & 4.86\\
  UsbKbDxe & 13.94 & 13.28 & 4.71\\
  DriverHealthManagerDxe & 15.12 & 14.42 & 4.65\\
  UhciDxe & 14.50 & 13.84 & 4.58\\
  ConSplitterDxe & 18.81 & 17.98 & 4.40\\
  EhciDxe & 17.00 & 16.26 & 4.37\\
  UsbBusDxe & 16.88 & 16.15 & 4.31\\
  PciSioSerialDxe & 17.06 & 16.34 & 4.26\\
  PlatformDxe & 13.69 & 13.11 & 4.22\\
  VlanConfigDxe & 18.06 & 17.33 & 4.07\\
  TerminalDxe & 17.56 & 16.85 & 4.05\\
  QemuVideoDxe & 17.88 & 17.17 & 3.93\\
  AcpiTableDxe & 16.88 & 16.24 & 3.75\\
  LinuxInitrdDynamicShellCommand & 19.81 & 19.07 & 3.75\\
  SnpDxe & 18.12 & 17.46 & 3.66\\
  PciHostBridgeDxe & 17.81 & 17.23 & 3.25\\
  PeiCore & 24.69 & 23.89 & 3.23\\
  Udp4Dxe & 23.50 & 22.79 & 3.03\\
  AtaAtapiPassThruDxe & 26.19 & 25.40 & 3.01\\
  MnpDxe & 23.94 & 23.23 & 2.97\\
  DevicePathDxe & 36.69 & 35.61 & 2.94\\
  Mtftp4Dxe & 25.94 & 25.18 & 2.92\\
  Fat & 23.25 & 22.58 & 2.89\\
  NvmExpressDxe & 23.94 & 23.25 & 2.87\\
  CpuMpPei (MpInitLibUp) & 27.88 & 27.09 & 2.83\\
  VirtioFsDxe & 21.44 & 20.85 & 2.73\\
  tftpDynamicCommand & 29.12 & 28.35 & 2.66\\
  Dhcp4Dxe & 27.88 & 27.15 & 2.61\\
  XhciDxe & 26.69 & 26.02 & 2.52\\
  PlatformPei & 28.94 & 28.21 & 2.51\\
  RamDiskDxe & 27.81 & 27.14 & 2.42\\
  ScsiDisk & 29.69 & 29.00 & 2.32\\
  VariableRuntimeDxe & 30.62 & 29.92 & 2.30\\
  httpDynamicCommand & 37.00 & 36.23 & 2.07\\
  SecMain & 37.56 & 36.80 & 2.02\\
  BootScriptExecutorDxe & 39.75 & 38.98 & 1.95\\
  CpuMpPei & 42.94 & 42.11 & 1.93\\
  PciBusDxe & 39.69 & 38.99 & 1.75\\
  TcpDxe & 43.19 & 42.46 & 1.68\\
  UefiPxeBcDxe & 56.31 & 55.58 & 1.30\\
  DisplayEngine & 56.88 & 56.16 & 1.26\\
  Ip4Dxe & 60.94 & 60.22 & 1.18\\
  SetupBrowser & 64.81 & 64.12 & 1.07\\
  BdsDxe & 71.69 & 70.94 & 1.05\\
  HiiDatabase & 83.75 & 83.02 & 0.87\\
  DxeCore & 132.81 & 131.68 & 0.85\\
  IScsiDxe & 99.88 & 99.09 & 0.79\\
  UiApp & 109.38 & 108.63 & 0.68\\
  Shell & 826.81 & 822.01 & 0.58\\
  \midrule
  \textbf{Average} & 27.60 & 26.87 & 9.53\\
  \textbf{Total} & 2,897.81 & 2,821.46 & 2.63\\
  \bottomrule
\end{longtable}

\begin{longtable}{l c c c}
  \caption{OVMF (X64, CLANGDWARF) PE-to-UE Size Comparison.}\\\\
  \toprule
  \multirow{2}{*}[-2pt]{\textbf{Module}} & \multicolumn{2}{c}{\textbf{Size} [KiB]} & \multirow{2}{*}[-2pt]{\textbf{Saving} [\%]}\\
  \cmidrule{2-3}
  & \textbf{PE} & \textbf{UE} &\\
  \midrule
  WatchdogTimer & 1.44 & 0.75 & 47.83\\
  StatusCodeHandlerPei & 1.06 & 0.55 & 47.79\\
  MonotonicCounterRuntimeDxe & 1.56 & 0.93 & 40.50\\
  DpcDxe & 1.50 & 0.92 & 38.54\\
  NullMemoryTestDxe & 1.56 & 0.98 & 37.00\\
  CapsuleRuntimeDxe & 1.56 & 1.02 & 34.50\\
  EnglishDxe & 2.06 & 1.35 & 34.47\\
  ReportStatusCodeRouterPei & 1.75 & 1.16 & 33.48\\
  IncompatiblePciDeviceSupportDxe & 1.69 & 1.13 & 32.87\\
  StatusCodeHandlerRuntimeDxe & 1.62 & 1.15 & 29.33\\
  Metronome & 2.12 & 1.54 & 27.57\\
  BootGraphicsResourceTableDxe & 2.75 & 2.05 & 25.57\\
  EmuVariableFvbRuntimeDxe & 2.69 & 2.05 & 23.55\\
  SecurityStubDxe & 2.62 & 2.02 & 23.21\\
  VirtioPciDeviceDxe & 3.69 & 2.86 & 22.46\\
  LocalApicTimerDxe & 3.69 & 2.95 & 20.13\\
  ResetSystemRuntimeDxe & 4.06 & 3.34 & 17.88\\
  CpuIo2Dxe & 4.00 & 3.31 & 17.19\\
  SataController & 3.94 & 3.28 & 16.67\\
  ReportStatusCodeRouterRuntimeDxe & 3.62 & 3.05 & 15.73\\
  VirtioRngDxe & 4.00 & 3.39 & 15.23\\
  SioBusDxe & 4.31 & 3.66 & 15.22\\
  RuntimeDxe & 4.25 & 3.62 & 14.71\\
  VirtioBlkDxe & 4.94 & 4.27 & 13.61\\
  SmbiosPlatformDxe & 4.56 & 3.95 & 13.36\\
  QemuKernelLoaderFsDxe & 6.00 & 5.20 & 13.28\\
  S3Resume2Pei & 6.19 & 5.40 & 12.75\\
  ConPlatformDxe & 5.94 & 5.20 & 12.50\\
  AmdSevDxe & 5.00 & 4.38 & 12.50\\
  TdxDxe & 6.31 & 5.60 & 11.26\\
  VirtioScsiDxe & 6.25 & 5.56 & 11.00\\
  PcdPeim & 7.38 & 6.57 & 10.91\\
  DiskIoDxe & 6.62 & 5.95 & 10.26\\
  QemuRamfbDxe & 7.44 & 6.68 & 10.19\\
  DxeIpl & 6.88 & 6.18 & 10.11\\
  Virtio10 & 7.75 & 6.97 & 10.08\\
  FvbServicesRuntimeDxe & 7.00 & 6.35 & 9.26\\
  ScsiBus & 6.75 & 6.12 & 9.26\\
  PcdDxe & 9.19 & 8.40 & 8.59\\
  PcRtc & 7.19 & 6.59 & 8.37\\
  SmbiosDxe & 9.06 & 8.35 & 7.84\\
  UsbMassStorageDxe & 9.81 & 9.05 & 7.72\\
  EbcDxe & 12.12 & 11.26 & 7.15\\
  VirtioNetDxe & 9.94 & 9.24 & 7.00\\
  IoMmuDxe & 9.50 & 8.84 & 6.99\\
  ArpDxe & 10.62 & 9.92 & 6.62\\
  Ps2KeyboardDxe & 11.81 & 11.09 & 6.15\\
  PartitionDxe & 11.75 & 11.03 & 6.12\\
  AtaBusDxe & 12.06 & 11.34 & 5.96\\
  S3SaveStateDxe & 11.31 & 10.64 & 5.94\\
  QemuFwCfgAcpiPlatform & 10.25 & 9.66 & 5.79\\
  PciHotPlugInitDxe & 7.81 & 7.39 & 5.40\\
  FaultTolerantWriteDxe & 9.56 & 9.05 & 5.39\\
  LogoDxe & 12.94 & 12.25 & 5.31\\
  VirtioGpuDxe & 13.69 & 12.98 & 5.14\\
  DriverHealthManagerDxe & 14.94 & 14.17 & 5.13\\
  UhciDxe & 13.88 & 13.17 & 5.07\\
  UdfDxe & 12.69 & 12.05 & 4.99\\
  UsbBusDxe & 15.56 & 14.80 & 4.87\\
  UsbKbDxe & 13.00 & 12.37 & 4.87\\
  GraphicsConsoleDxe & 14.44 & 13.75 & 4.76\\
  AcpiTableDxe & 15.94 & 15.19 & 4.71\\
  TerminalDxe & 16.81 & 16.02 & 4.69\\
  ConSplitterDxe & 17.38 & 16.57 & 4.63\\
  PciSioSerialDxe & 16.50 & 15.80 & 4.26\\
  PlatformDxe & 13.88 & 13.30 & 4.17\\
  QemuVideoDxe & 18.19 & 17.44 & 4.12\\
  EhciDxe & 16.19 & 15.54 & 4.01\\
  SnpDxe & 18.00 & 17.30 & 3.91\\
  VlanConfigDxe & 17.38 & 16.70 & 3.87\\
  PeiCore & 26.81 & 25.83 & 3.67\\
  LinuxInitrdDynamicShellCommand & 18.19 & 17.53 & 3.61\\
  PciHostBridgeDxe & 17.44 & 16.82 & 3.54\\
  DevicePathDxe & 34.44 & 33.23 & 3.49\\
  AtaAtapiPassThruDxe & 24.62 & 23.80 & 3.36\\
  Fat & 22.06 & 21.34 & 3.26\\
  NvmExpressDxe & 23.00 & 22.28 & 3.12\\
  Mtftp4Dxe & 24.75 & 23.99 & 3.06\\
  Dhcp4Dxe & 26.19 & 25.41 & 2.98\\
  Udp4Dxe & 22.69 & 22.02 & 2.96\\
  CpuMpPei (MpInitLibUp) & 26.62 & 25.84 & 2.93\\
  MnpDxe & 23.38 & 22.71 & 2.84\\
  VirtioFsDxe & 20.88 & 20.33 & 2.62\\
  XhciDxe & 25.75 & 25.09 & 2.58\\
  RamDiskDxe & 27.00 & 26.30 & 2.58\\
  PlatformPei & 27.94 & 27.23 & 2.52\\
  tftpDynamicCommand & 27.62 & 26.96 & 2.40\\
  VariableRuntimeDxe & 29.56 & 28.88 & 2.30\\
  httpDynamicCommand & 35.94 & 35.18 & 2.11\\
  ScsiDisk & 29.56 & 28.95 & 2.06\\
  BootScriptExecutorDxe & 42.44 & 41.57 & 2.04\\
  SecMain & 42.06 & 41.20 & 2.04\\
  CpuMpPei & 41.31 & 40.48 & 2.02\\
  PciBusDxe & 37.19 & 36.49 & 1.87\\
  TcpDxe & 41.00 & 40.26 & 1.81\\
  UefiPxeBcDxe & 53.12 & 52.38 & 1.40\\
  Ip4Dxe & 56.75 & 55.97 & 1.38\\
  SetupBrowser & 63.62 & 62.82 & 1.26\\
  DisplayEngine & 58.00 & 57.27 & 1.25\\
  BdsDxe & 68.25 & 67.52 & 1.06\\
  HiiDatabase & 82.19 & 81.41 & 0.95\\
  DxeCore & 121.44 & 120.42 & 0.84\\
  IScsiDxe & 94.06 & 93.35 & 0.76\\
  UiApp & 104.00 & 103.25 & 0.72\\
  Shell & 806.62 & 801.78 & 0.60\\
  \midrule
  \textbf{Average} & 26.63 & 25.89 & 10.09\\
  \textbf{Total} & 2,796.44 & 2,718.62 & 2.78\\
  \bottomrule
\end{longtable}

\begin{longtable}{l c c c}
  \caption{ArmVirtQemu (ARM, GCC5) PE-to-UE Size Comparison.}\\\\
  \toprule
  \multirow{2}{*}[-2pt]{\textbf{Module}} & \multicolumn{2}{c}{\textbf{Size} [KiB]} & \multirow{2}{*}[-2pt]{\textbf{Saving} [\%]}\\
  \cmidrule{2-3}
  & \textbf{PE} & \textbf{UE} &\\
  \midrule
  WatchdogTimer & 1.38 & 0.70 & 49.43\\
  DpcDxe & 1.31 & 0.71 & 45.83\\
  HighMemDxe & 1.31 & 0.74 & 43.45\\
  EnglishDxe & 1.75 & 1.07 & 38.84\\
  CapsuleRuntimeDxe & 1.50 & 0.93 & 38.02\\
  ResetSystemRuntimeDxe & 2.12 & 1.52 & 28.68\\
  SecurityStubDxe & 2.62 & 1.99 & 24.11\\
  VirtioPciDeviceDxe & 3.00 & 2.37 & 21.09\\
  ReportStatusCodeRouterRuntimeDxe & 2.94 & 2.36 & 19.68\\
  MetronomeDxe & 2.94 & 2.36 & 19.68\\
  MonotonicCounterRuntimeDxe & 3.50 & 2.82 & 19.42\\
  CpuPei & 3.38 & 2.79 & 17.36\\
  SerialDxe & 3.69 & 3.06 & 16.95\\
  VirtioRngDxe & 3.94 & 3.30 & 16.07\\
  PlatformHasAcpiDtDxe & 3.75 & 3.15 & 16.04\\
  ArmTimerDxe & 4.00 & 3.41 & 14.65\\
  ArmPciCpuIo2Dxe & 4.44 & 3.82 & 13.91\\
  VirtioFdtDxe & 4.69 & 4.07 & 13.17\\
  SmbiosPlatformDxe & 4.81 & 4.18 & 13.15\\
  ConPlatformDxe & 4.75 & 4.19 & 11.84\\
  FdtClientDxe & 4.75 & 4.19 & 11.84\\
  MemoryInit & 5.69 & 5.05 & 11.26\\
  QemuKernelLoaderFsDxe & 6.00 & 5.34 & 11.07\\
  RuntimeDxe & 6.00 & 5.37 & 10.55\\
  RealTimeClock & 5.56 & 4.98 & 10.39\\
  VirtioScsiDxe & 5.44 & 4.88 & 10.20\\
  PlatformPei & 5.25 & 4.73 & 9.82\\
  VirtioBlkDxe & 6.19 & 5.60 & 9.47\\
  ArmGicDxe & 6.12 & 5.56 & 9.18\\
  ScsiBus & 6.81 & 6.20 & 9.06\\
  Virtio10 & 7.69 & 6.99 & 9.04\\
  DiskIoDxe & 7.31 & 6.67 & 8.76\\
  ArpDxe & 8.12 & 7.45 & 8.37\\
  PciHotPlugInitDxe & 7.62 & 7.02 & 7.99\\
  QemuRamfbDxe & 7.00 & 6.47 & 7.59\\
  SmbiosDxe & 8.19 & 7.58 & 7.44\\
  FaultTolerantWriteDxe & 8.12 & 7.52 & 7.40\\
  VirtioNetDxe & 8.12 & 7.53 & 7.31\\
  VirtNorFlashDxe & 8.75 & 8.13 & 7.05\\
  UsbMassStorageDxe & 10.00 & 9.37 & 6.33\\
  UsbBusDxe & 10.88 & 10.23 & 5.96\\
  PartitionDxe & 12.25 & 11.52 & 5.93\\
  PcdDxe & 9.44 & 8.88 & 5.88\\
  ArmPlatformPrePeiCore & 11.44 & 10.77 & 5.81\\
  UsbKbDxe & 10.19 & 9.60 & 5.75\\
  TerminalDxe & 12.19 & 11.51 & 5.58\\
  UdfDxe & 11.88 & 11.25 & 5.26\\
  GraphicsConsoleDxe & 12.25 & 11.62 & 5.17\\
  VirtioGpuDxe & 12.62 & 11.99 & 5.01\\
  PlatformDxe & 13.31 & 12.66 & 4.87\\
  UhciDxe & 11.25 & 10.71 & 4.79\\
  EhciDxe & 13.50 & 12.88 & 4.57\\
  DxeIpl & 12.56 & 11.99 & 4.54\\
  DriverHealthManagerDxe & 13.69 & 13.09 & 4.34\\
  PciHostBridgeDxe & 16.06 & 15.38 & 4.23\\
  LogoDxe & 12.75 & 12.21 & 4.23\\
  ConSplitterDxe & 14.88 & 14.25 & 4.20\\
  VlanConfigDxe & 15.62 & 15.00 & 4.00\\
  Fat & 18.19 & 17.52 & 3.69\\
  VirtioFsDxe & 16.06 & 15.48 & 3.65\\
  LinuxInitrdDynamicShellCommand & 17.50 & 16.88 & 3.57\\
  MnpDxe & 18.94 & 18.30 & 3.38\\
  NvmExpressDxe & 19.56 & 18.91 & 3.31\\
  PeiCore & 20.62 & 19.95 & 3.26\\
  XhciDxe & 19.75 & 19.15 & 3.05\\
  Mtftp4Dxe & 20.06 & 19.45 & 3.04\\
  Dhcp4Dxe & 20.44 & 19.83 & 2.98\\
  Udp4Dxe & 18.88 & 18.31 & 2.98\\
  RamDiskDxe & 23.94 & 23.23 & 2.94\\
  VariableRuntimeDxe & 23.69 & 23.02 & 2.80\\
  ScsiDisk & 23.81 & 23.20 & 2.56\\
  PciBusDxe & 28.88 & 28.16 & 2.49\\
  tftpDynamicCommand & 24.50 & 23.89 & 2.49\\
  DevicePathDxe & 28.62 & 27.97 & 2.29\\
  TcpDxe & 31.56 & 30.86 & 2.23\\
  httpDynamicCommand & 30.19 & 29.55 & 2.10\\
  Ip4Dxe & 39.69 & 38.99 & 1.75\\
  SetupBrowser & 44.56 & 43.82 & 1.67\\
  UefiPxeBcDxe & 39.69 & 39.03 & 1.65\\
  DisplayEngine & 44.38 & 43.66 & 1.62\\
  BdsDxe & 43.81 & 43.15 & 1.52\\
  HiiDatabase & 54.00 & 53.28 & 1.33\\
  DxeCore & 74.12 & 73.40 & 0.98\\
  UiApp & 81.00 & 80.31 & 0.85\\
  Shell & 675.56 & 674.38 & 0.17\\
  \midrule
  \textbf{Average} & 22.70 & 22.06 & 9.60\\
  \textbf{Total} & 1,929.31 & 1,875.41 & 2.79\\
  \bottomrule
\end{longtable}

\begin{longtable}{l c c c}
  \caption{ArmVirtQemu (ARM, CLANGDWARF) PE-to-UE Size Comparison.}\\\\
  \toprule
  \multirow{2}{*}[-2pt]{\textbf{Module}} & \multicolumn{2}{c}{\textbf{Size} [KiB]} & \multirow{2}{*}[-2pt]{\textbf{Saving} [\%]}\\
  \cmidrule{2-3}
  & \textbf{PE} & \textbf{UE} &\\
  \midrule
  WatchdogTimer & 1.31 & 0.65 & 50.60\\
  HighMemDxe & 1.31 & 0.73 & 44.05\\
  CapsuleRuntimeDxe & 1.56 & 0.92 & 41.00\\
  DpcDxe & 1.56 & 0.98 & 37.50\\
  EnglishDxe & 2.12 & 1.45 & 31.99\\
  ResetSystemRuntimeDxe & 2.38 & 1.74 & 26.64\\
  SecurityStubDxe & 2.56 & 1.95 & 23.78\\
  ReportStatusCodeRouterRuntimeDxe & 3.25 & 2.60 & 19.95\\
  MetronomeDxe & 2.94 & 2.37 & 19.41\\
  MonotonicCounterRuntimeDxe & 3.38 & 2.73 & 18.98\\
  CpuPei & 3.50 & 2.84 & 18.75\\
  VirtioPciDeviceDxe & 3.31 & 2.71 & 18.16\\
  PlatformHasAcpiDtDxe & 3.69 & 3.06 & 16.95\\
  VirtioRngDxe & 4.25 & 3.59 & 15.44\\
  SerialDxe & 3.62 & 3.07 & 15.30\\
  ArmTimerDxe & 4.38 & 3.80 & 13.21\\
  VirtioFdtDxe & 4.88 & 4.26 & 12.66\\
  SmbiosPlatformDxe & 5.31 & 4.64 & 12.65\\
  RealTimeClock & 5.31 & 4.66 & 12.35\\
  ArmPciCpuIo2Dxe & 4.88 & 4.33 & 11.22\\
  ConPlatformDxe & 5.31 & 4.73 & 11.03\\
  FdtClientDxe & 5.88 & 5.24 & 10.77\\
  VirtioScsiDxe & 6.06 & 5.45 & 10.05\\
  PlatformPei & 5.75 & 5.21 & 9.38\\
  QemuKernelLoaderFsDxe & 6.62 & 6.02 & 9.08\\
  VirtioBlkDxe & 6.56 & 5.97 & 9.05\\
  ArmGicDxe & 6.62 & 6.05 & 8.61\\
  ScsiBus & 7.00 & 6.45 & 7.92\\
  MemoryInit & 7.94 & 7.34 & 7.58\\
  DiskIoDxe & 8.25 & 7.63 & 7.48\\
  RuntimeDxe & 7.88 & 7.29 & 7.44\\
  QemuRamfbDxe & 7.81 & 7.24 & 7.30\\
  PciHotPlugInitDxe & 8.56 & 7.95 & 7.12\\
  Virtio10 & 9.62 & 8.95 & 6.98\\
  VirtioNetDxe & 8.69 & 8.09 & 6.83\\
  ArpDxe & 9.88 & 9.22 & 6.65\\
  VirtNorFlashDxe & 10.88 & 10.20 & 6.25\\
  FaultTolerantWriteDxe & 11.94 & 11.27 & 5.56\\
  GraphicsConsoleDxe & 13.00 & 12.34 & 5.11\\
  PartitionDxe & 13.50 & 12.82 & 5.03\\
  UsbMassStorageDxe & 12.19 & 11.58 & 5.00\\
  PcdDxe & 12.25 & 11.65 & 4.91\\
  TerminalDxe & 14.94 & 14.24 & 4.65\\
  PlatformDxe & 15.06 & 14.39 & 4.46\\
  UsbKbDxe & 14.00 & 13.38 & 4.41\\
  UdfDxe & 14.00 & 13.39 & 4.35\\
  LogoDxe & 12.75 & 12.20 & 4.35\\
  VirtioGpuDxe & 14.25 & 13.64 & 4.28\\
  DriverHealthManagerDxe & 15.94 & 15.27 & 4.17\\
  SmbiosDxe & 16.00 & 15.34 & 4.10\\
  DxeIpl & 15.44 & 14.84 & 3.85\\
  UhciDxe & 16.88 & 16.26 & 3.66\\
  VlanConfigDxe & 17.81 & 17.16 & 3.64\\
  ConSplitterDxe & 18.75 & 18.09 & 3.54\\
  UsbBusDxe & 16.31 & 15.76 & 3.40\\
  ArmPlatformPrePeiCore & 19.50 & 18.86 & 3.29\\
  LinuxInitrdDynamicShellCommand & 18.12 & 17.56 & 3.10\\
  Udp4Dxe & 23.44 & 22.71 & 3.10\\
  NvmExpressDxe & 22.38 & 21.69 & 3.07\\
  VirtioFsDxe & 20.06 & 19.46 & 3.00\\
  PciHostBridgeDxe & 24.25 & 23.57 & 2.80\\
  EhciDxe & 23.25 & 22.60 & 2.79\\
  Dhcp4Dxe & 24.56 & 23.89 & 2.74\\
  MnpDxe & 23.50 & 22.89 & 2.59\\
  tftpDynamicCommand & 27.50 & 26.80 & 2.56\\
  Fat & 25.88 & 25.22 & 2.54\\
  ScsiDisk & 27.81 & 27.12 & 2.50\\
  Mtftp4Dxe & 25.75 & 25.12 & 2.46\\
  RamDiskDxe & 27.25 & 26.65 & 2.21\\
  httpDynamicCommand & 35.31 & 34.59 & 2.04\\
  PeiCore & 35.38 & 34.68 & 1.97\\
  VariableRuntimeDxe & 34.94 & 34.29 & 1.86\\
  XhciDxe & 34.50 & 33.89 & 1.77\\
  DevicePathDxe & 44.00 & 43.24 & 1.72\\
  TcpDxe & 44.12 & 43.40 & 1.65\\
  PciBusDxe & 40.38 & 39.72 & 1.63\\
  Ip4Dxe & 55.19 & 54.44 & 1.36\\
  UefiPxeBcDxe & 48.62 & 47.98 & 1.33\\
  SetupBrowser & 64.69 & 63.90 & 1.22\\
  BdsDxe & 61.50 & 60.76 & 1.21\\
  DisplayEngine & 59.06 & 58.36 & 1.19\\
  HiiDatabase & 67.81 & 67.02 & 1.16\\
  DxeCore & 88.06 & 87.27 & 0.90\\
  UiApp & 111.75 & 110.97 & 0.70\\
  Shell & 778.69 & 777.34 & 0.17\\
  \midrule
  \textbf{Average} & 28.20 & 27.55 & 8.70\\
  \textbf{Total} & 2,397.06 & 2,341.73 & 2.31\\
  \bottomrule
\end{longtable}

\begin{longtable}{l c c c}
  \caption{ArmVirtQemu (AARCH64, GCC5) PE-to-UE Size Comparison.}\\\\
  \toprule
  \multirow{2}{*}[-2pt]{\textbf{Module}} & \multicolumn{2}{c}{\textbf{Size} [KiB]} & \multirow{2}{*}[-2pt]{\textbf{Saving} [\%]}\\
  \cmidrule{2-3}
  & \textbf{PE} & \textbf{UE} &\\
  \midrule
  MetronomeDxe & 1.06 & 0.48 & 54.41\\
  WatchdogTimer & 1.44 & 0.81 & 43.48\\
  DpcDxe & 1.75 & 1.14 & 34.82\\
  EnglishDxe & 2.19 & 1.46 & 33.21\\
  HighMemDxe & 1.50 & 1.02 & 32.29\\
  MonotonicCounterRuntimeDxe & 1.81 & 1.24 & 31.47\\
  CapsuleRuntimeDxe & 1.69 & 1.16 & 31.02\\
  CpuPei & 1.81 & 1.35 & 25.43\\
  ArmTimerDxe & 2.62 & 1.96 & 25.30\\
  ArmPciCpuIo2Dxe & 2.94 & 2.21 & 24.73\\
  ResetSystemRuntimeDxe & 2.94 & 2.23 & 23.94\\
  PlatformHasAcpiDtDxe & 3.00 & 2.30 & 23.44\\
  SerialDxe & 3.81 & 3.05 & 19.88\\
  PlatformPei & 4.56 & 3.81 & 16.44\\
  BootGraphicsResourceTableDxe & 4.00 & 3.35 & 16.21\\
  SecurityStubDxe & 4.06 & 3.42 & 15.77\\
  VirtioFdtDxe & 4.50 & 3.80 & 15.45\\
  SmbiosPlatformDxe & 4.62 & 3.93 & 15.03\\
  RealTimeClock & 4.00 & 3.42 & 14.45\\
  ReportStatusCodeRouterRuntimeDxe & 4.19 & 3.60 & 13.99\\
  VirtioPciDeviceDxe & 4.88 & 4.20 & 13.78\\
  RuntimeDxe & 4.44 & 3.83 & 13.73\\
  QemuKernelLoaderFsDxe & 5.94 & 5.23 & 11.97\\
  VirtioRngDxe & 5.69 & 5.02 & 11.81\\
  VirtioBlkDxe & 5.81 & 5.16 & 11.16\\
  ArmGicDxe & 6.56 & 5.88 & 10.36\\
  QemuRamfbDxe & 7.25 & 6.52 & 10.13\\
  ScsiBus & 7.56 & 6.82 & 9.81\\
  ConPlatformDxe & 6.94 & 6.30 & 9.23\\
  VirtNorFlashDxe & 7.56 & 6.87 & 9.19\\
  FdtClientDxe & 7.94 & 7.22 & 9.06\\
  DiskIoDxe & 7.81 & 7.12 & 8.80\\
  QemuFwCfgAcpiPlatform & 9.31 & 8.53 & 8.39\\
  VirtioScsiDxe & 7.88 & 7.23 & 8.13\\
  PciHotPlugInitDxe & 8.25 & 7.59 & 7.95\\
  SmbiosDxe & 9.88 & 9.16 & 7.28\\
  MemoryInit & 7.25 & 6.75 & 6.90\\
  PcdDxe & 12.25 & 11.52 & 5.99\\
  UsbMassStorageDxe & 12.44 & 11.70 & 5.90\\
  VirtioNetDxe & 12.00 & 11.30 & 5.79\\
  Virtio10 & 13.44 & 12.68 & 5.64\\
  ArpDxe & 13.06 & 12.38 & 5.20\\
  PartitionDxe & 13.44 & 12.74 & 5.17\\
  LogoDxe & 13.06 & 12.43 & 4.84\\
  UsbKbDxe & 14.62 & 13.95 & 4.59\\
  UdfDxe & 14.44 & 13.80 & 4.44\\
  FaultTolerantWriteDxe & 11.00 & 10.52 & 4.40\\
  DxeIpl & 15.62 & 14.95 & 4.35\\
  PciHostBridgeDxe & 17.31 & 16.56 & 4.33\\
  UhciDxe & 15.56 & 14.89 & 4.32\\
  GraphicsConsoleDxe & 16.38 & 15.68 & 4.25\\
  ConSplitterDxe & 20.50 & 19.64 & 4.19\\
  VirtioGpuDxe & 15.31 & 14.67 & 4.18\\
  DriverHealthManagerDxe & 17.00 & 16.30 & 4.09\\
  TerminalDxe & 18.06 & 17.34 & 4.02\\
  UsbBusDxe & 18.19 & 17.47 & 3.95\\
  VlanConfigDxe & 19.62 & 18.89 & 3.74\\
  EhciDxe & 17.81 & 17.16 & 3.64\\
  AcpiTableDxe & 18.25 & 17.59 & 3.64\\
  PlatformDxe & 15.75 & 15.22 & 3.37\\
  LinuxInitrdDynamicShellCommand & 21.31 & 20.63 & 3.19\\
  PeiCore & 27.62 & 26.84 & 2.86\\
  DevicePathDxe & 39.50 & 38.39 & 2.81\\
  NvmExpressDxe & 25.25 & 24.55 & 2.75\\
  Udp4Dxe & 25.88 & 25.16 & 2.75\\
  Fat & 24.94 & 24.27 & 2.66\\
  VirtioFsDxe & 21.06 & 20.51 & 2.63\\
  Mtftp4Dxe & 27.31 & 26.59 & 2.63\\
  MnpDxe & 25.94 & 25.27 & 2.56\\
  Dhcp4Dxe & 29.62 & 28.87 & 2.56\\
  tftpDynamicCommand & 30.38 & 29.63 & 2.44\\
  XhciDxe & 27.81 & 27.18 & 2.28\\
  ScsiDisk & 31.25 & 30.55 & 2.25\\
  RamDiskDxe & 29.88 & 29.22 & 2.20\\
  httpDynamicCommand & 37.69 & 36.93 & 2.01\\
  VariableRuntimeDxe & 35.12 & 34.47 & 1.87\\
  ArmPlatformPrePeiCore & 37.75 & 37.07 & 1.80\\
  TcpDxe & 44.81 & 44.04 & 1.73\\
  PciBusDxe & 42.50 & 41.78 & 1.69\\
  UefiPxeBcDxe & 53.88 & 53.14 & 1.36\\
  Ip4Dxe & 57.56 & 56.88 & 1.18\\
  DisplayEngine & 58.38 & 57.69 & 1.18\\
  SetupBrowser & 67.25 & 66.48 & 1.14\\
  BdsDxe & 67.75 & 67.02 & 1.08\\
  DxeCore & 105.69 & 104.59 & 1.04\\
  HiiDatabase & 80.31 & 79.55 & 0.94\\
  UiApp & 107.50 & 106.73 & 0.72\\
  Shell & 809.38 & 804.54 & 0.60\\
  \midrule
  \textbf{Average} & 28.73 & 27.99 & 9.52\\
  \textbf{Total} & 2,527.88 & 2,463.16 & 2.56\\
  \bottomrule
\end{longtable}

\begin{longtable}{l c c c}
  \caption{ArmVirtQemu (AARCH64, CLANGDWARF) PE-to-UE Size Comparison.}\\\\
  \toprule
  \multirow{2}{*}[-2pt]{\textbf{Module}} & \multicolumn{2}{c}{\textbf{Size} [KiB]} & \multirow{2}{*}[-2pt]{\textbf{Saving} [\%]}\\
  \cmidrule{2-3}
  & \textbf{PE} & \textbf{UE} &\\
  \midrule
  MetronomeDxe & 1.12 & 0.49 & 56.25\\
  WatchdogTimer & 1.44 & 0.80 & 44.02\\
  HighMemDxe & 1.31 & 0.84 & 35.71\\
  DpcDxe & 1.81 & 1.20 & 33.62\\
  CapsuleRuntimeDxe & 1.69 & 1.15 & 31.94\\
  MonotonicCounterRuntimeDxe & 1.94 & 1.32 & 31.85\\
  CpuPei & 1.56 & 1.07 & 31.50\\
  EnglishDxe & 2.62 & 1.86 & 29.17\\
  ResetSystemRuntimeDxe & 2.62 & 1.98 & 24.70\\
  PlatformHasAcpiDtDxe & 3.12 & 2.38 & 23.75\\
  ArmTimerDxe & 2.88 & 2.24 & 22.01\\
  SerialDxe & 3.62 & 2.91 & 19.83\\
  ArmPciCpuIo2Dxe & 3.62 & 2.92 & 19.40\\
  BootGraphicsResourceTableDxe & 4.00 & 3.27 & 18.16\\
  VirtioFdtDxe & 4.44 & 3.69 & 16.90\\
  SecurityStubDxe & 3.75 & 3.15 & 16.04\\
  SmbiosPlatformDxe & 4.50 & 3.78 & 15.97\\
  ReportStatusCodeRouterRuntimeDxe & 4.25 & 3.62 & 14.89\\
  PlatformPei & 5.19 & 4.44 & 14.46\\
  VirtioPciDeviceDxe & 4.94 & 4.23 & 14.40\\
  RealTimeClock & 4.00 & 3.45 & 13.87\\
  VirtioRngDxe & 5.50 & 4.84 & 11.93\\
  VirtioBlkDxe & 5.88 & 5.20 & 11.44\\
  RuntimeDxe & 6.31 & 5.62 & 10.89\\
  ConPlatformDxe & 7.50 & 6.77 & 9.79\\
  ScsiBus & 7.62 & 6.89 & 9.63\\
  ArmGicDxe & 7.25 & 6.55 & 9.59\\
  QemuKernelLoaderFsDxe & 7.75 & 7.01 & 9.58\\
  QemuRamfbDxe & 7.88 & 7.17 & 8.93\\
  FdtClientDxe & 9.06 & 8.30 & 8.45\\
  DiskIoDxe & 8.81 & 8.10 & 8.07\\
  VirtioScsiDxe & 8.50 & 7.82 & 8.00\\
  QemuFwCfgAcpiPlatform & 11.38 & 10.59 & 6.94\\
  Virtio10 & 11.38 & 10.62 & 6.59\\
  VirtNorFlashDxe & 10.50 & 9.81 & 6.55\\
  MemoryInit & 9.56 & 8.95 & 6.37\\
  PciHotPlugInitDxe & 9.38 & 8.78 & 6.33\\
  VirtioNetDxe & 12.50 & 11.79 & 5.69\\
  LogoDxe & 12.94 & 12.27 & 5.13\\
  UsbMassStorageDxe & 14.44 & 13.72 & 4.98\\
  PartitionDxe & 15.50 & 14.78 & 4.64\\
  PcdDxe & 16.94 & 16.16 & 4.57\\
  GraphicsConsoleDxe & 17.56 & 16.84 & 4.09\\
  UsbKbDxe & 19.00 & 18.23 & 4.07\\
  ArpDxe & 16.19 & 15.54 & 4.01\\
  TerminalDxe & 21.44 & 20.63 & 3.75\\
  FaultTolerantWriteDxe & 14.56 & 14.02 & 3.70\\
  UdfDxe & 20.19 & 19.45 & 3.64\\
  VirtioGpuDxe & 17.69 & 17.05 & 3.58\\
  SmbiosDxe & 20.69 & 19.96 & 3.51\\
  DriverHealthManagerDxe & 21.31 & 20.60 & 3.34\\
  PciHostBridgeDxe & 23.50 & 22.75 & 3.19\\
  UsbBusDxe & 24.31 & 23.54 & 3.18\\
  DxeIpl & 21.81 & 21.13 & 3.12\\
  ConSplitterDxe & 26.69 & 25.87 & 3.07\\
  AcpiTableDxe & 24.00 & 23.27 & 3.03\\
  UhciDxe & 22.50 & 21.82 & 3.02\\
  LinuxInitrdDynamicShellCommand & 24.75 & 24.05 & 2.84\\
  VlanConfigDxe & 26.94 & 26.20 & 2.73\\
  NvmExpressDxe & 29.06 & 28.38 & 2.37\\
  EhciDxe & 29.44 & 28.77 & 2.28\\
  PlatformDxe & 25.06 & 24.52 & 2.15\\
  Dhcp4Dxe & 36.06 & 35.29 & 2.14\\
  MnpDxe & 32.25 & 31.56 & 2.13\\
  Mtftp4Dxe & 37.38 & 36.62 & 2.03\\
  Fat & 35.19 & 34.48 & 2.02\\
  PeiCore & 47.12 & 46.22 & 1.92\\
  ArmPlatformPrePeiCore & 46.69 & 45.80 & 1.91\\
  ScsiDisk & 38.25 & 37.52 & 1.90\\
  tftpDynamicCommand & 36.69 & 35.99 & 1.90\\
  Udp4Dxe & 33.00 & 32.38 & 1.89\\
  VirtioFsDxe & 29.00 & 28.45 & 1.89\\
  RamDiskDxe & 38.69 & 38.02 & 1.72\\
  httpDynamicCommand & 47.19 & 46.45 & 1.56\\
  XhciDxe & 48.19 & 47.52 & 1.39\\
  DevicePathDxe & 90.56 & 89.34 & 1.35\\
  PciBusDxe & 56.38 & 55.71 & 1.18\\
  VariableRuntimeDxe & 49.75 & 49.16 & 1.18\\
  TcpDxe & 63.69 & 62.98 & 1.10\\
  UefiPxeBcDxe & 71.56 & 70.77 & 1.10\\
  DxeCore & 130.88 & 129.77 & 0.85\\
  Ip4Dxe & 88.56 & 87.87 & 0.79\\
  DisplayEngine & 102.19 & 101.42 & 0.75\\
  BdsDxe & 94.56 & 93.88 & 0.73\\
  SetupBrowser & 112.88 & 112.18 & 0.62\\
  HiiDatabase & 148.12 & 147.33 & 0.54\\
  UiApp & 168.94 & 168.16 & 0.46\\
  Shell & 1,125.56 & 1,120.91 & 0.41\\
  \midrule
  \textbf{Average} & 40.08 & 39.33 & 9.01\\
  \textbf{Total} & 3,527.00 & 3,460.97 & 1.87\\
  \bottomrule
\end{longtable}


\FloatBarrier
\chapter{Implementation Fuzz-Testing Code Coverage}
\label{chap:code_cov}

This appendix contains the code coverage report data for \gls{fuzz-testing} the \glsxtrshort{UE} parsing and conversion code. It was generated by LCOV~\cite{lcov} and is publicly available and reproducible for artifact evaluation~\cite{thesis-git}.

\begin{table}[!htb]
  \centering
  \caption{UE Parsing Fuzz-Testing Coverage.}
  \hfill\\
  \begin{tabular}{l c c c}
    \toprule
    \multirow{2}{*}[-2pt]{\textbf{File}} & \multicolumn{3}{c}{\textbf{Coverage} [\%]}\\
    \cmidrule{2-4}
    & \textbf{Lines} & \textbf{Functions} & \textbf{Branches}\\
    \midrule
    TestUe/Ue.c & 100 & 100 & 100\\
    TestUefiImage/UefiImage.c & 100 & 100 & 100\\
    BaseUeImageLib/UeImageLib.c & 100 & 100 & 100\\
    BaseUefiImageLib/CommonSupport.c & 100 & 100 & 100\\
    BaseUefiImageLib/UeSupport.c & 100 & 100 & 100\\
    \midrule
    \textbf{Total} & 100 & 100 & 100\\
    \bottomrule
  \end{tabular}
\end{table}

\begin{table}[!htb]
  \centering
  \caption{UE Conversion Fuzz-Testing Coverage.}
  \hfill\\
  \begin{tabular}{l c c c}
    \toprule
    \multirow{2}{*}[-2pt]{\textbf{File}} & \multicolumn{3}{c}{\textbf{Coverage} [\%]}\\
    \cmidrule{2-4}
    & \textbf{Lines} & \textbf{Functions} & \textbf{Branches}\\
    \midrule
    TestEmit/Emit.c & 100 & 100 & 100\\
    ImageTool/DynamicBuffer.c & 94.4 & 100 & 88.9\\
    ImageTool/Image.c & 99.1 & 100 & 98.9\\
    ImageTool/ImageToolEmit.c & 100 & 100 & 100\\
    ImageTool/PeScan.c & 95.3 & 100 & 94.3\\
    ImageTool/UeEmit.c & 97.8 & 100 & 96.5\\
    ImageTool/UeScan.c & 100 & 100 & 100\\
    ImageTool/UefiImageScan.c & 98.1 & 100 & 97.6\\
    \midrule
    \textbf{Total} & 98.1 & 100 & 97.3\\
    \bottomrule
  \end{tabular}
\end{table}

